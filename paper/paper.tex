\documentclass[a4paper]{article}

% Packages
\usepackage{graphicx}
\usepackage[margin=1in]{geometry}
\usepackage[backend=bibtex]{biblatex}
\usepackage{comment}


\addbibresource{references.bib}

\title{Evaluation of Machine Translation in Languages}
\author{Stefan Liemawan Adji}
\date{\today}

\begin{document}

\maketitle

\section{Objectives}

This paper aims to evaluate existing techniques and circumstances around the task of machine translation across different languages.

\section{Related Work}

\section{Methodology}

Take the list of most spoken language by population according to Wikipedia \cite{wikipedia-list-languages}, then take one for each branch.

\begin{table}[htbp]
    \small
    \centering
    \begin{tabular}{|l|l|r|l|l|}
        \hline
        \textbf{No.} & \textbf{Language} & \textbf{Native speakers} & \textbf{Language family} & \textbf{Branch}   \\
                     &                   & \textbf{(in millions)}   &                          &                   \\
        \hline
        1            & Egyptian Arabic   & 78                       & Afroasiatic              & Semitic           \\
        2            & German            & 76                       & Indo-European            & Germanic          \\
        3            & Hausa             & 54                       & Afroasiatic              & Chadic            \\
        4            & Hindi             & 345                      & Indo-European            & Indo-Aryan        \\
        6            & Japanese          & 123                      & Japonic                  & Japanese          \\
        7            & Javanese          & 68                       & Austronesian             & Malayo-Polynesian \\
        8            & Korean            & 81                       & Koreanic                 & —                 \\
        9            & Mandarin Chinese  & 941                      & Sino-Tibetan             & Sinitic           \\
        5            & Persian           & 62                       & Indo-European            & Iranian           \\
        10           & Russian           & 148                      & Indo-European            & Balto-Slavic      \\
        11           & Spanish           & 486                      & Indo-European            & Romance           \\
        13           & Telugu            & 83                       & Dravidian                & South-Central     \\
        12           & Turkish           & 84                       & Turkic                   & Oghuz             \\
        14           & Vietnamese        & 85                       & Austroasiatic            & Vietic            \\
        \hline
    \end{tabular}
    \caption{List of most spoken languages per branch}
\end{table}

Egyptian Arabic, Javanese, and Telugu might be discarded due to low examples in dataset.

\printbibliography
\end{document}
